% Options for packages loaded elsewhere
\PassOptionsToPackage{unicode}{hyperref}
\PassOptionsToPackage{hyphens}{url}
%
\documentclass[
]{article}
\usepackage{lmodern}
\usepackage{amsmath}
\usepackage{ifxetex,ifluatex}
\ifnum 0\ifxetex 1\fi\ifluatex 1\fi=0 % if pdftex
  \usepackage[T1]{fontenc}
  \usepackage[utf8]{inputenc}
  \usepackage{textcomp} % provide euro and other symbols
  \usepackage{amssymb}
\else % if luatex or xetex
  \usepackage{unicode-math}
  \defaultfontfeatures{Scale=MatchLowercase}
  \defaultfontfeatures[\rmfamily]{Ligatures=TeX,Scale=1}
\fi
% Use upquote if available, for straight quotes in verbatim environments
\IfFileExists{upquote.sty}{\usepackage{upquote}}{}
\IfFileExists{microtype.sty}{% use microtype if available
  \usepackage[]{microtype}
  \UseMicrotypeSet[protrusion]{basicmath} % disable protrusion for tt fonts
}{}
\makeatletter
\@ifundefined{KOMAClassName}{% if non-KOMA class
  \IfFileExists{parskip.sty}{%
    \usepackage{parskip}
  }{% else
    \setlength{\parindent}{0pt}
    \setlength{\parskip}{6pt plus 2pt minus 1pt}}
}{% if KOMA class
  \KOMAoptions{parskip=half}}
\makeatother
\usepackage{xcolor}
\IfFileExists{xurl.sty}{\usepackage{xurl}}{} % add URL line breaks if available
\IfFileExists{bookmark.sty}{\usepackage{bookmark}}{\usepackage{hyperref}}
\hypersetup{
  pdftitle={ENV 790.30 - Time Series Analysis for Energy Data \textbar{} Spring 2021},
  pdfauthor={Xueying Feng},
  hidelinks,
  pdfcreator={LaTeX via pandoc}}
\urlstyle{same} % disable monospaced font for URLs
\usepackage[margin=2.54cm]{geometry}
\usepackage{color}
\usepackage{fancyvrb}
\newcommand{\VerbBar}{|}
\newcommand{\VERB}{\Verb[commandchars=\\\{\}]}
\DefineVerbatimEnvironment{Highlighting}{Verbatim}{commandchars=\\\{\}}
% Add ',fontsize=\small' for more characters per line
\usepackage{framed}
\definecolor{shadecolor}{RGB}{248,248,248}
\newenvironment{Shaded}{\begin{snugshade}}{\end{snugshade}}
\newcommand{\AlertTok}[1]{\textcolor[rgb]{0.94,0.16,0.16}{#1}}
\newcommand{\AnnotationTok}[1]{\textcolor[rgb]{0.56,0.35,0.01}{\textbf{\textit{#1}}}}
\newcommand{\AttributeTok}[1]{\textcolor[rgb]{0.77,0.63,0.00}{#1}}
\newcommand{\BaseNTok}[1]{\textcolor[rgb]{0.00,0.00,0.81}{#1}}
\newcommand{\BuiltInTok}[1]{#1}
\newcommand{\CharTok}[1]{\textcolor[rgb]{0.31,0.60,0.02}{#1}}
\newcommand{\CommentTok}[1]{\textcolor[rgb]{0.56,0.35,0.01}{\textit{#1}}}
\newcommand{\CommentVarTok}[1]{\textcolor[rgb]{0.56,0.35,0.01}{\textbf{\textit{#1}}}}
\newcommand{\ConstantTok}[1]{\textcolor[rgb]{0.00,0.00,0.00}{#1}}
\newcommand{\ControlFlowTok}[1]{\textcolor[rgb]{0.13,0.29,0.53}{\textbf{#1}}}
\newcommand{\DataTypeTok}[1]{\textcolor[rgb]{0.13,0.29,0.53}{#1}}
\newcommand{\DecValTok}[1]{\textcolor[rgb]{0.00,0.00,0.81}{#1}}
\newcommand{\DocumentationTok}[1]{\textcolor[rgb]{0.56,0.35,0.01}{\textbf{\textit{#1}}}}
\newcommand{\ErrorTok}[1]{\textcolor[rgb]{0.64,0.00,0.00}{\textbf{#1}}}
\newcommand{\ExtensionTok}[1]{#1}
\newcommand{\FloatTok}[1]{\textcolor[rgb]{0.00,0.00,0.81}{#1}}
\newcommand{\FunctionTok}[1]{\textcolor[rgb]{0.00,0.00,0.00}{#1}}
\newcommand{\ImportTok}[1]{#1}
\newcommand{\InformationTok}[1]{\textcolor[rgb]{0.56,0.35,0.01}{\textbf{\textit{#1}}}}
\newcommand{\KeywordTok}[1]{\textcolor[rgb]{0.13,0.29,0.53}{\textbf{#1}}}
\newcommand{\NormalTok}[1]{#1}
\newcommand{\OperatorTok}[1]{\textcolor[rgb]{0.81,0.36,0.00}{\textbf{#1}}}
\newcommand{\OtherTok}[1]{\textcolor[rgb]{0.56,0.35,0.01}{#1}}
\newcommand{\PreprocessorTok}[1]{\textcolor[rgb]{0.56,0.35,0.01}{\textit{#1}}}
\newcommand{\RegionMarkerTok}[1]{#1}
\newcommand{\SpecialCharTok}[1]{\textcolor[rgb]{0.00,0.00,0.00}{#1}}
\newcommand{\SpecialStringTok}[1]{\textcolor[rgb]{0.31,0.60,0.02}{#1}}
\newcommand{\StringTok}[1]{\textcolor[rgb]{0.31,0.60,0.02}{#1}}
\newcommand{\VariableTok}[1]{\textcolor[rgb]{0.00,0.00,0.00}{#1}}
\newcommand{\VerbatimStringTok}[1]{\textcolor[rgb]{0.31,0.60,0.02}{#1}}
\newcommand{\WarningTok}[1]{\textcolor[rgb]{0.56,0.35,0.01}{\textbf{\textit{#1}}}}
\usepackage{graphicx}
\makeatletter
\def\maxwidth{\ifdim\Gin@nat@width>\linewidth\linewidth\else\Gin@nat@width\fi}
\def\maxheight{\ifdim\Gin@nat@height>\textheight\textheight\else\Gin@nat@height\fi}
\makeatother
% Scale images if necessary, so that they will not overflow the page
% margins by default, and it is still possible to overwrite the defaults
% using explicit options in \includegraphics[width, height, ...]{}
\setkeys{Gin}{width=\maxwidth,height=\maxheight,keepaspectratio}
% Set default figure placement to htbp
\makeatletter
\def\fps@figure{htbp}
\makeatother
\setlength{\emergencystretch}{3em} % prevent overfull lines
\providecommand{\tightlist}{%
  \setlength{\itemsep}{0pt}\setlength{\parskip}{0pt}}
\setcounter{secnumdepth}{-\maxdimen} % remove section numbering
\usepackage{enumerate}
\usepackage{enumitem}
\ifluatex
  \usepackage{selnolig}  % disable illegal ligatures
\fi

\title{ENV 790.30 - Time Series Analysis for Energy Data \textbar{}
Spring 2021}
\usepackage{etoolbox}
\makeatletter
\providecommand{\subtitle}[1]{% add subtitle to \maketitle
  \apptocmd{\@title}{\par {\large #1 \par}}{}{}
}
\makeatother
\subtitle{Assignment 5 - Due date 03/12/21}
\author{Xueying Feng}
\date{}

\begin{document}
\maketitle

\hypertarget{directions}{%
\subsection{Directions}\label{directions}}

You should open the .rmd file corresponding to this assignment on
RStudio. The file is available on our class repository on Github. And to
do so you will need to fork our repository and link it to your RStudio.

Once you have the project open the first thing you will do is change
``Student Name'' on line 3 with your name. Then you will start working
through the assignment by \textbf{creating code and output} that answer
each question. Be sure to use this assignment document. Your report
should contain the answer to each question and any plots/tables you
obtained (when applicable).

When you have completed the assignment, \textbf{Knit} the text and code
into a single PDF file. Rename the pdf file such that it includes your
first and last name (e.g., ``LuanaLima\_TSA\_A05\_Sp21.Rmd''). Submit
this pdf using Sakai.

\hypertarget{questions}{%
\subsection{Questions}\label{questions}}

This assignment has general questions about ARIMA Models.

Packages needed for this assignment: ``forecast'',``tseries''. Do not
forget to load them before running your script, since they are NOT
default packages.\textbackslash{}

\begin{Shaded}
\begin{Highlighting}[]
\CommentTok{\#Load/install required package here}
\FunctionTok{library}\NormalTok{(lubridate)}
\end{Highlighting}
\end{Shaded}

\begin{verbatim}
## 
## Attaching package: 'lubridate'
\end{verbatim}

\begin{verbatim}
## The following objects are masked from 'package:base':
## 
##     date, intersect, setdiff, union
\end{verbatim}

\begin{Shaded}
\begin{Highlighting}[]
\FunctionTok{library}\NormalTok{(ggplot2)}
\FunctionTok{library}\NormalTok{(forecast)  }
\end{Highlighting}
\end{Shaded}

\begin{verbatim}
## Registered S3 method overwritten by 'quantmod':
##   method            from
##   as.zoo.data.frame zoo
\end{verbatim}

\begin{Shaded}
\begin{Highlighting}[]
\FunctionTok{library}\NormalTok{(Kendall)}
\FunctionTok{library}\NormalTok{(tseries)}
\FunctionTok{library}\NormalTok{(outliers)}
\FunctionTok{library}\NormalTok{(tidyverse)}
\end{Highlighting}
\end{Shaded}

\begin{verbatim}
## -- Attaching packages --------------------------------------- tidyverse 1.3.0 --
\end{verbatim}

\begin{verbatim}
## v tibble  3.1.0     v dplyr   1.0.5
## v tidyr   1.1.3     v stringr 1.4.0
## v readr   1.4.0     v forcats 0.5.1
## v purrr   0.3.4
\end{verbatim}

\begin{verbatim}
## -- Conflicts ------------------------------------------ tidyverse_conflicts() --
## x lubridate::as.difftime() masks base::as.difftime()
## x lubridate::date()        masks base::date()
## x dplyr::filter()          masks stats::filter()
## x lubridate::intersect()   masks base::intersect()
## x dplyr::lag()             masks stats::lag()
## x lubridate::setdiff()     masks base::setdiff()
## x lubridate::union()       masks base::union()
\end{verbatim}

\hypertarget{q1}{%
\subsection{Q1}\label{q1}}

Describe the important characteristics of the sample autocorrelation
function (ACF) plot and the partial sample autocorrelation function
(PACF) plot for the following models:

\begin{enumerate}
\def\labelenumi{(\alph{enumi})}
\tightlist
\item
  AR(2)
\end{enumerate}

\begin{quote}
Answer: The stationary series has positive autocorrelation at lag 1 in
ACF, and the ACF exponentially decays to 0 as the lag increases; Order
should see PACF, cutoff count from Lag 1
\end{quote}

\begin{enumerate}
\def\labelenumi{(\alph{enumi})}
\setcounter{enumi}{1}
\tightlist
\item
  MA(1)
\end{enumerate}

\begin{quote}
Answer: The stationary series has negative autocorrelation at lag 1 in
ACF; Order should see ACF, cutoff count from Lag1
\end{quote}

\hypertarget{q2}{%
\subsection{Q2}\label{q2}}

Recall that the non-seasonal ARIMA is described by three parameters
ARIMA\((p,d,q)\) where \(p\) is the order of the autoregressive
component, \(d\) is the number of times the series need to be
differenced to obtain stationarity and \(q\) is the order of the moving
average component. If we don't need to difference the series, we don't
need to specify the ``I'' part and we can use the short version, i.e.,
the ARMA\((p,q)\). Consider three models: ARMA(1,0), ARMA(0,1) and
ARMA(1,1) with parameters \(\phi=0.6\) and \(\theta= 0.9\). The \(\phi\)
refers to the AR coefficient and the \(\theta\) refers to the MA
coefficient. Use R to generate \(n=100\) observations from each of these
three models

\begin{Shaded}
\begin{Highlighting}[]
\FunctionTok{set.seed}\NormalTok{(}\DecValTok{123}\NormalTok{)}
\NormalTok{ARMA11 }\OtherTok{\textless{}{-}} \FunctionTok{arima.sim}\NormalTok{(}\AttributeTok{model=}\FunctionTok{list}\NormalTok{(}\AttributeTok{ar=}\NormalTok{.}\DecValTok{6}\NormalTok{,}\AttributeTok{ma=}\NormalTok{.}\DecValTok{9}\NormalTok{),}\AttributeTok{n=}\DecValTok{100}\NormalTok{)}
\NormalTok{ARMA01 }\OtherTok{\textless{}{-}} \FunctionTok{arima.sim}\NormalTok{(}\AttributeTok{model=}\FunctionTok{list}\NormalTok{(}\AttributeTok{ma=}\NormalTok{.}\DecValTok{9}\NormalTok{),}\AttributeTok{n=}\DecValTok{100}\NormalTok{)}
\NormalTok{ARMA10 }\OtherTok{\textless{}{-}} \FunctionTok{arima.sim}\NormalTok{(}\AttributeTok{model=}\FunctionTok{list}\NormalTok{(}\AttributeTok{ar=}\NormalTok{.}\DecValTok{6}\NormalTok{),}\AttributeTok{n=}\DecValTok{100}\NormalTok{)}
\end{Highlighting}
\end{Shaded}

\begin{enumerate}
\def\labelenumi{(\alph{enumi})}
\tightlist
\item
  Plot the sample ACF for each of these models in one window to
  facilitate comparison (Hint: use command par(mfrow=c(1,3)) that
  divides the plotting window in three columns).
\end{enumerate}

\begin{Shaded}
\begin{Highlighting}[]
\FunctionTok{par}\NormalTok{(}\AttributeTok{mar=}\FunctionTok{c}\NormalTok{(}\DecValTok{3}\NormalTok{,}\DecValTok{3}\NormalTok{,}\DecValTok{3}\NormalTok{,}\DecValTok{0}\NormalTok{));}\FunctionTok{par}\NormalTok{(}\AttributeTok{mfrow=}\FunctionTok{c}\NormalTok{(}\DecValTok{1}\NormalTok{,}\DecValTok{3}\NormalTok{))}
\FunctionTok{Acf}\NormalTok{(ARMA11, }\AttributeTok{lag.max =} \ConstantTok{NULL}\NormalTok{, }\AttributeTok{plot =} \ConstantTok{TRUE}\NormalTok{)}
\FunctionTok{Acf}\NormalTok{(ARMA01, }\AttributeTok{lag.max =} \ConstantTok{NULL}\NormalTok{, }\AttributeTok{plot =} \ConstantTok{TRUE}\NormalTok{)}
\FunctionTok{Acf}\NormalTok{(ARMA10, }\AttributeTok{lag.max =} \ConstantTok{NULL}\NormalTok{, }\AttributeTok{plot =} \ConstantTok{TRUE}\NormalTok{)}
\end{Highlighting}
\end{Shaded}

\includegraphics{XueyingFeng_TSA_A05_Sp21_files/figure-latex/unnamed-chunk-3-1.pdf}

\begin{enumerate}
\def\labelenumi{(\alph{enumi})}
\setcounter{enumi}{1}
\tightlist
\item
  Plot the sample PACF for each of these models in one window to
  facilitate comparison.
\end{enumerate}

\begin{Shaded}
\begin{Highlighting}[]
\FunctionTok{par}\NormalTok{(}\AttributeTok{mar=}\FunctionTok{c}\NormalTok{(}\DecValTok{3}\NormalTok{,}\DecValTok{3}\NormalTok{,}\DecValTok{3}\NormalTok{,}\DecValTok{0}\NormalTok{));}\FunctionTok{par}\NormalTok{(}\AttributeTok{mfrow=}\FunctionTok{c}\NormalTok{(}\DecValTok{1}\NormalTok{,}\DecValTok{3}\NormalTok{))}
\FunctionTok{Pacf}\NormalTok{(ARMA11, }\AttributeTok{lag.max =} \ConstantTok{NULL}\NormalTok{, }\AttributeTok{plot =} \ConstantTok{TRUE}\NormalTok{)}
\FunctionTok{Pacf}\NormalTok{(ARMA01, }\AttributeTok{lag.max =} \ConstantTok{NULL}\NormalTok{, }\AttributeTok{plot =} \ConstantTok{TRUE}\NormalTok{)}
\FunctionTok{Pacf}\NormalTok{(ARMA10, }\AttributeTok{lag.max =} \ConstantTok{NULL}\NormalTok{, }\AttributeTok{plot =} \ConstantTok{TRUE}\NormalTok{)}
\end{Highlighting}
\end{Shaded}

\includegraphics{XueyingFeng_TSA_A05_Sp21_files/figure-latex/unnamed-chunk-4-1.pdf}

\begin{enumerate}
\def\labelenumi{(\alph{enumi})}
\setcounter{enumi}{2}
\tightlist
\item
  Look at the ACFs and PACFs. Imagine you had these plots for a data set
  and you were asked to identify the model, i.e., is it AR, MA or ARMA
  and the order of each component. Would you be identify them correctly?
  Explain your answer.
\end{enumerate}

\begin{quote}
Answer: I would not identify them correctly. I may identify ARMA11 plot
as AR process, because this series has positive autocorrelation at lag
1, and shows decay pattern in ACF plot. However, both ACF and PACF show
slow decay. Hence, the ARMA (1,1) model would also be appropriate for
the series. So I should experiment with both ARMA (1,1) and AR for the
process and later select the optimal model. Base on plot, order = 4
ARMA01 pacf shows slow decay, but no nagetive autocorrelation at lag 1
in ACF, so it is hard to tell it is a MA process. ARMA10 ACF plot has
positive autocorrelation at lag 1 and ACF shows sharp decay, AR terms
work best, and order = 1.
\end{quote}

\begin{enumerate}
\def\labelenumi{(\alph{enumi})}
\setcounter{enumi}{3}
\tightlist
\item
  Compare the ACF and PACF values R computed with the theoretical values
  you provided for the coefficients. Do they match? Explain your answer.
\end{enumerate}

\begin{quote}
Answer: I can only tell the phi in AR plot (third plot), which almost
equals to theoretical values phi=0.6.
\end{quote}

\begin{enumerate}
\def\labelenumi{(\alph{enumi})}
\setcounter{enumi}{4}
\tightlist
\item
  Increase number of observations to \(n=1000\) and repeat parts
  (a)-(d).
\end{enumerate}

\begin{Shaded}
\begin{Highlighting}[]
\FunctionTok{set.seed}\NormalTok{(}\DecValTok{123}\NormalTok{)}
\NormalTok{ARMA11 }\OtherTok{\textless{}{-}} \FunctionTok{arima.sim}\NormalTok{(}\AttributeTok{model=}\FunctionTok{list}\NormalTok{(}\AttributeTok{ar=}\NormalTok{.}\DecValTok{6}\NormalTok{,}\AttributeTok{ma=}\NormalTok{.}\DecValTok{9}\NormalTok{),}\AttributeTok{n=}\DecValTok{1000}\NormalTok{)}
\NormalTok{ARMA01 }\OtherTok{\textless{}{-}} \FunctionTok{arima.sim}\NormalTok{(}\AttributeTok{model=}\FunctionTok{list}\NormalTok{(}\AttributeTok{ma=}\NormalTok{.}\DecValTok{9}\NormalTok{),}\AttributeTok{n=}\DecValTok{1000}\NormalTok{)}
\NormalTok{ARMA10 }\OtherTok{\textless{}{-}} \FunctionTok{arima.sim}\NormalTok{(}\AttributeTok{model=}\FunctionTok{list}\NormalTok{(}\AttributeTok{ar=}\NormalTok{.}\DecValTok{6}\NormalTok{),}\AttributeTok{n=}\DecValTok{1000}\NormalTok{)}

\CommentTok{\#ACF}
\FunctionTok{par}\NormalTok{(}\AttributeTok{mar=}\FunctionTok{c}\NormalTok{(}\DecValTok{3}\NormalTok{,}\DecValTok{3}\NormalTok{,}\DecValTok{3}\NormalTok{,}\DecValTok{0}\NormalTok{));}\FunctionTok{par}\NormalTok{(}\AttributeTok{mfrow=}\FunctionTok{c}\NormalTok{(}\DecValTok{1}\NormalTok{,}\DecValTok{3}\NormalTok{))}
\FunctionTok{Acf}\NormalTok{(ARMA11, }\AttributeTok{lag.max =} \ConstantTok{NULL}\NormalTok{, }\AttributeTok{plot =} \ConstantTok{TRUE}\NormalTok{)}
\FunctionTok{Acf}\NormalTok{(ARMA01, }\AttributeTok{lag.max =} \ConstantTok{NULL}\NormalTok{, }\AttributeTok{plot =} \ConstantTok{TRUE}\NormalTok{)}
\FunctionTok{Acf}\NormalTok{(ARMA10, }\AttributeTok{lag.max =} \ConstantTok{NULL}\NormalTok{, }\AttributeTok{plot =} \ConstantTok{TRUE}\NormalTok{)}
\end{Highlighting}
\end{Shaded}

\includegraphics{XueyingFeng_TSA_A05_Sp21_files/figure-latex/unnamed-chunk-5-1.pdf}

\begin{Shaded}
\begin{Highlighting}[]
\CommentTok{\#PACF}
\FunctionTok{par}\NormalTok{(}\AttributeTok{mar=}\FunctionTok{c}\NormalTok{(}\DecValTok{3}\NormalTok{,}\DecValTok{3}\NormalTok{,}\DecValTok{3}\NormalTok{,}\DecValTok{0}\NormalTok{));}\FunctionTok{par}\NormalTok{(}\AttributeTok{mfrow=}\FunctionTok{c}\NormalTok{(}\DecValTok{1}\NormalTok{,}\DecValTok{3}\NormalTok{))}
\FunctionTok{Pacf}\NormalTok{(ARMA11, }\AttributeTok{lag.max =} \ConstantTok{NULL}\NormalTok{, }\AttributeTok{plot =} \ConstantTok{TRUE}\NormalTok{)}
\FunctionTok{Pacf}\NormalTok{(ARMA01, }\AttributeTok{lag.max =} \ConstantTok{NULL}\NormalTok{, }\AttributeTok{plot =} \ConstantTok{TRUE}\NormalTok{)}
\FunctionTok{Pacf}\NormalTok{(ARMA10, }\AttributeTok{lag.max =} \ConstantTok{NULL}\NormalTok{, }\AttributeTok{plot =} \ConstantTok{TRUE}\NormalTok{)}
\end{Highlighting}
\end{Shaded}

\includegraphics{XueyingFeng_TSA_A05_Sp21_files/figure-latex/unnamed-chunk-5-2.pdf}
\textgreater{} When I increased to 1000 obsevation, the pattern is very
similar as 100 obsevation.

\hypertarget{q3}{%
\subsection{Q3}\label{q3}}

Consider the ARIMA model
\(y_t=0.7*y_{t-1}-0.25*y_{t-12}+a_t-0.1*a_{t-1}\)

\begin{enumerate}
\def\labelenumi{(\alph{enumi})}
\tightlist
\item
  Identify the model using the notation ARIMA\((p,d,q)(P,D,Q)_ s\),
  i.e., identify the integers \(p,d,q,P,D,Q,s\) (if possible) from the
  equation.
\end{enumerate}

\begin{quote}
p = 1, d = 0, q = 1, P = 1, D = 0, Q = 0, s = 12 ARIMA(1,0,1)(1,0,0)12
\end{quote}

\begin{enumerate}
\def\labelenumi{(\alph{enumi})}
\setcounter{enumi}{1}
\tightlist
\item
  Also from the equation what are the values of the parameters, i.e.,
  model coefficients.
\end{enumerate}

\begin{quote}
phi1 = 0.7, phi12 = -0.25 and theta = -0.1
\end{quote}

\hypertarget{q4}{%
\subsection{Q4}\label{q4}}

Plot the ACF and PACF of a seasonal ARIMA\((0, 1)\times(1, 0)_{12}\)
model with \(\phi =0 .8\) and \(\theta = 0.5\) using R. The \(12\) after
the bracket tells you that \(s=12\), i.e., the seasonal lag is 12,
suggesting monthly data whose behavior is repeated every 12 months. You
can generate as many observations as you like. Note the Integrated part
was omitted. It means the series do not need differencing, therefore
\(d=D=0\). Plot ACF and PACF for the simulated data. Comment if the
plots are well representing the model you simulated, i.e., would you be
able to identify the order of both non-seasonal and seasonal components
from the plots? Explain.

\begin{Shaded}
\begin{Highlighting}[]
\CommentTok{\#install.packages(sarima)}
\CommentTok{\#install.packages(glmnet)}
\CommentTok{\#install.packages(bestglm)}

\CommentTok{\#library(glmnet)}
\CommentTok{\#library(bestglm)}
\FunctionTok{library}\NormalTok{(sarima)}
\end{Highlighting}
\end{Shaded}

\begin{verbatim}
## Loading required package: FitAR
\end{verbatim}

\begin{verbatim}
## Loading required package: lattice
\end{verbatim}

\begin{verbatim}
## Loading required package: leaps
\end{verbatim}

\begin{verbatim}
## Loading required package: ltsa
\end{verbatim}

\begin{verbatim}
## Loading required package: bestglm
\end{verbatim}

\begin{verbatim}
## 
## Attaching package: 'FitAR'
\end{verbatim}

\begin{verbatim}
## The following object is masked from 'package:forecast':
## 
##     BoxCox
\end{verbatim}

\begin{verbatim}
## Loading required package: stats4
\end{verbatim}

\begin{Shaded}
\begin{Highlighting}[]
\FunctionTok{require}\NormalTok{(}\StringTok{"PolynomF"}\NormalTok{)}
\end{Highlighting}
\end{Shaded}

\begin{verbatim}
## Loading required package: PolynomF
\end{verbatim}

\begin{verbatim}
## 
## Attaching package: 'PolynomF'
\end{verbatim}

\begin{verbatim}
## The following object is masked from 'package:purrr':
## 
##     zap
\end{verbatim}

\begin{Shaded}
\begin{Highlighting}[]
\CommentTok{\# yt = phi ∗ yt−12 + at − theta ∗ at−1}

\NormalTok{ARIMAModel }\OtherTok{\textless{}{-}} \FunctionTok{sim\_sarima}\NormalTok{(}\AttributeTok{n=}\DecValTok{1000}\NormalTok{, }\AttributeTok{model =} \FunctionTok{list}\NormalTok{(}\AttributeTok{ar=}\FloatTok{0.8}\NormalTok{, }\AttributeTok{ma=}\FloatTok{0.5}\NormalTok{, }\AttributeTok{sar=}\DecValTok{0}\NormalTok{, }\AttributeTok{sma=}\DecValTok{0}\NormalTok{, }\AttributeTok{iorder=}\DecValTok{0}\NormalTok{, }\AttributeTok{siorder=}\DecValTok{0}\NormalTok{, }\AttributeTok{nseasons=}\DecValTok{12}\NormalTok{))}

\FunctionTok{par}\NormalTok{(}\AttributeTok{mar=}\FunctionTok{c}\NormalTok{(}\DecValTok{3}\NormalTok{,}\DecValTok{3}\NormalTok{,}\DecValTok{3}\NormalTok{,}\DecValTok{0}\NormalTok{));}\FunctionTok{par}\NormalTok{(}\AttributeTok{mfrow=}\FunctionTok{c}\NormalTok{(}\DecValTok{1}\NormalTok{,}\DecValTok{2}\NormalTok{))}
\FunctionTok{Acf}\NormalTok{(ARIMAModel, }\AttributeTok{lag.max =} \ConstantTok{NULL}\NormalTok{)}
\FunctionTok{Pacf}\NormalTok{(ARIMAModel, }\AttributeTok{lag.max =} \ConstantTok{NULL}\NormalTok{)}
\end{Highlighting}
\end{Shaded}

\includegraphics{XueyingFeng_TSA_A05_Sp21_files/figure-latex/unnamed-chunk-6-1.pdf}

\end{document}
